\documentclass[12pt, letter paper]{report}
\usepackage{graphicx} % Required for inserting images
\usepackage[document]{ragged2e}
\usepackage[a4paper, total={6in, 7in}]{geometry}
\setcounter{tocdepth}{6} 
\setcounter{secnumdepth}{6} 
\setcounter{chapter}{0}
\usepackage{setspace}
\usepackage{blindtext}
\usepackage{layout}
\usepackage[dvipsnames]{xcolor}
\usepackage{biblatex} %Imports biblatex package
\addbibresource{sample.bib} %Import the bibliography file
\usepackage{geometry}
\geometry{
 a4paper,
 total={170mm,257mm},
 left=27mm,
 top=20mm,
 right=27mm,
 }
\title{DBMS MINIPROJECT REPORT }
\author{Akshay}
\date{August 2023}

\begin{document}
\thispagestyle{empty}
%\maketitle
%cover page 
\begin{center}
\Large\textbf{ ST JOSEPH ENGINEERING COLLEGE\\ }
\textbf{An Autonomous Institution}\\
Affiliated to VTU, Belagavi\\
\textbf{Mangaluru-575028}
\end{center}

\begin{figure}[h]
 \centering
 \includegraphics[scale=1.1]{sjec.jpeg}
 \label{sjeclogo}
\end{figure}

\begin{center}
   \textbf{ MINI PROJECT REPORT ON }
\end{center}
\begin{center}
   \Large\textbf{“TAILOR'S DATABASE”}
\end{center}



 
\begin{center}
   \Large \textbf{Submitted By \\}
\end{center}

\begin{center}
\justifying{
\hspace{3.5 cm}Abhik L Salian\hspace{3.5cm}4SO21CS004\\
%\vspace{0.5 cm}
%XXXXXX\hspace{9.5 cm}4SO21CS403
}
\justifying{

\hspace{2.9 cm}Akshaya Kumar S\hspace{2.9 cm}4SO21CS014
}
\end{center}
   \begin{center}
\large \textbf{\\Under the guidance of \\}
\large  \textbf{ \\Ms Pruthvi  M R\\ }
\large  Assistant Professor,\\ Department of CSE
\end{center}

\vspace{0.5cm}
\begin{center}
\large\textbf{\\ DEPARTMENT OF COMPUTER SCIENCE AND ENGINEERING\\ 2022-2023}

\end{center}
\vspace{2cm}
\thispagestyle{empty}
% certificate Page 
\section*{}
\begin{center}
\Large\textbf{\\ ST JOSEPH ENGINEERING COLLEGE}\\
\end{center}
\begin{center}
\textbf{\textit{An Autonomous Institution}}\\

\large \textbf{Vamanjoor, Mangaluru-575028}\\
\large \textbf{ DEPARTMENT OF COMPUTER SCIENCE AND ENGINEERING} 
\end{center}
\vspace{0.1cm}
\begin{center}
\begin{figure}[h]
 \centering
 \includegraphics[scale=0.7]{sjec.jpeg}
 \label{sjeclogo}
\end{figure}
\end{center}
\vspace{-1cm}
\begin{center}
\large\textbf{\textcolor{blue}{CERTIFICATE}}
\end{center}
\begin{center}
\justifying{
\large Certified that the project work entitled \textbf{“Tailor's Database”} carried out by }
\end{center}
\begin{center}
\justifying{
\hspace{3.5 cm}Abhik L Salian\hspace{2.62 cm}4SO21CS004
}
\end{center}
\begin{center}
\justifying{
\hspace{3.5 cm}Akshaya Kumar S\hspace{2.1 cm}4SO21CS014
}
\end{center}

\begin{center}
\justifying{
\par{

\large bonafide students of IV semester students in partial fulfillment for the award of Bachelor of Engineering in Computer Science and Engineering of St Joseph Engineering College during the year 2022-23.It is certified that all corrections/suggestions indicated during Internal Evaluation have been incorporated in the report.The project report has been approved as it satisfies the academic requirements in respect of miniproject work .\\\\\\

} 
}
\end{center}
\begin{center}
\justifying{ 
\large\textbf{Ms Pruthvi M R \hspace{5 cm}  Dr Sridevi Saralaya  \\}
\hspace{1cm}\large\textbf{  Project Guide \hspace{6.7 cm} HOD-CSE 
}
}

\end{center}
\begin{center}
  
\large\textbf{ \\EXTERNAL VIVA}
\end{center}
\justifying{ 
\large\textbf{
NAME OF THE EXAMINER  \hspace{4 cm}  SIGNATURE \\\\ 1.\\\\2.}}

\thispagestyle{empty}
\newpage
\chapter*{\centering Acknowledgment}
\justifying{
\addcontentsline{toc}{chapter}{\numberline{}Acknowledgment}
% \linespread
  \large The satisfaction and euphoria that accompanies the successful completion of any task would be incomplete without mentioning the people who made it possible, whose constant guidance and encouragement crowned our efforts with success.\\
    \\
    We take this opportunity to thank those who have helped and motivated us throughout the completion of this project.\\
    \\
    We would like to express our deep and sincere gratitude to our project guide, \textbf{Ms Pruthvi M R}, Assistant Professor, Department of Computer Science and Engineering, for her constant guidance and support, without which this project wouldn’t have been completed successfully. \\
    \\ 
    We owe our great debt to \textbf{Dr Sridevi Saralaya}, Head of the Department of Computer Science and Engineering, for her support and encouragement during the course of development of this project. \\
    \\
    We are immensely grateful to our Principal, \textbf{Dr Rio D’Souza}, our Director, \textbf{Rev. Fr Wilfred Prakash D’Souza}, and Assistant Director \textbf{Rev. Fr Kenneth Rayner Crasta} for their support and encouragement. \\
    \\
    We extend our gratitude to the entire faculty and the staff of the Department of Computer Science and Engineering, SJEC, for their advice, kind co-operation and assistance throughout the academic year.\\
    \\
    Lastly, we would like to express our heartfelt appreciation towards our classmates and seniors for their guidance and suggestions.
 
    
 }
%/\thispagestyle{empty}
\pagenumbering{roman}
\chapter*{\centering Abstract}

\justifying{
\addcontentsline{toc}{chapter}{\numberline{}Abstract}
\begin{spacing}{1.5}
In the ever-evolving landscape of fashion and clothing customization, tailors and designers face the challenge of efficiently managing client measurements. This abstract introduces a Tailor's Database, a digital platform designed to empower tailors and clients alike in the process of storing and accessing personalized measurements.

The Tailor's Database is a secure and user-friendly system that offers a convenient and streamlined approach to recording, storing, and retrieving measurements for clients. It leverages technology to enhance the tailoring experience, ensuring accuracy, convenience, and personalization.

Tailors and clients often find themselves grappling with the complexities of managing measurements, but the Tailor's Database seeks to simplify this process. It adopts a client-centric approach, allowing clients to create profiles and securely store their measurements. This not only saves time during fittings but also fosters client loyalty by providing a seamless experience.

The system also facilitates effective communication between tailors and clients. Tailors can access the stored measurements, consult with clients remotely, and provide guidance, resulting in a collaborative and efficient tailoring process. This digital interaction enhances the tailoring experience, making it more engaging and responsive to individual preferences.

In conclusion, the Tailor's Database offers a comprehensive solution for tailors and clients seeking a modern and efficient way to manage measurements. By merging technology with the art of tailoring, it not only improves the accuracy and precision of garments but also enhances the overall client-tailor relationship. This digital platform represents a promising step forward in the ever-evolving world of personalized tailoring, ensuring both clients and tailors benefit from a more seamless and personalized experience.
\end{spacing}
%/\thispagestyle{empty}



%\section*{\centering TABLE OF CONTENT}
%/\thispagestyle{empty}
\renewcommand{\contentsname}{Table of Contents}
\tableofcontents
\addcontentsline{toc}{chapter}{\numberline{}Table of Contents}
\listoffigures
\addcontentsline{toc}{chapter}{\numberline{}List of Figures}
\listoftables
\addcontentsline{toc}{chapter}{\numberline{}List of Tables}
\newpage
%/\tableofcontents
\thispagestyle{empty}
\chapter{Introduction}
\begin{spacing}{1.5}
Amid the digital transformation sweeping industries, the tailoring sector is embracing innovation. This report details the creation of an online tailors' database and e-commerce platform, designed to meet the evolving needs of both tailors and customers. The platform simplifies the process of ordering custom shirts and pants while enabling users to securely manage their measurements.

Traditionally, tailoring has involved in-person visits, manual measurements, and lengthy wait times. This platform bridges tradition with modernity, incorporating user authentication, database management, e-commerce features, and secure payments.

The report offers insights into database design, web development, user interface, security, payment integration, testing, and maintenance. It emphasizes user-friendly design, data security, legal compliance, marketing, and support. This platform represents a forward-thinking approach to modernizing tailoring, offering convenience to users and efficiency to tailors.
\end{spacing}
\section{Problem Definition} 
\begin{spacing}{1.5}
To create a database that can store client’s data like client name, phone number, address, and all the measurements so that the tailor can use the data in the database without having to measure again and again.
\end{spacing}
%\subsection{vvv}
\section{Scope and Importance}
\subsection{Scope:}
\begin{spacing}{1.5}
\begin{itemize}
\item \textbf{Customer Management}: A tailor's DBMS can store and organize customer information, including measurements, preferences, contact details, and purchase history. This data can be used to offer personalized services and promotions.
\item \textbf{Order Processing}: The system can streamline the order management process, from taking measurements and fabric selection to tracking the progress of each order. This reduces errors and ensures timely deliveries.
\pagenumbering{arabic}
\item\textbf{Inventory Control}: Manage fabric and material inventory, ensuring availability and tracking usage for each order.
\item\textbf{Measurement Records}: Store and update customer measurements accurately, making it easy to retrieve and apply them to new orders.
\\
\\
\subsection{Importance:}
\item\textbf{Efficiency}: A DBMS streamlines operations by automating routine tasks, reducing manual errors, and improving the overall efficiency of the tailor shop.
\item\textbf{Customer Service}: The system enables tailors to provide personalized and consistent customer service by storing customer preferences and order histories.
\item\textbf{Accuracy}: Accurate measurement records and order tracking reduce the likelihood of errors in garment production, ensuring customer satisfaction.
\item \textbf{Data Security}: A DBMS can provide data security features to protect sensitive customer information, ensuring that it's not compromised.
\item \textbf{Competitive Advantage}: Implementing a modern DBMS can give tailors a competitive edge in the fashion industry by offering better services, faster delivery, and a more personalized customer experience.
\end{itemize}
\end{spacing}
\chapter{Software Requirement Specification}

\section{Functional Requirement Specification} 
Functional  Requirement Specification
\begin{itemize}
\item Details of operations conducted in every screen
\item Data handling logic should be entered into the system
\item It should have descriptions of system reports or other outputs
\item Complete information about the workflows performed by the system
\item It should clearly define who will be allowed to create/modify/delete the data in the system
\item How the system will fulfill applicable regulatory and compliance needs should be captured in the functional document.
\end{itemize}
\section{Software Requirement Specification}
Software Requirement Specification
\begin{itemize}
 \item {\textbf{Language} : HTML, CSS, Java Script, PHP}
 \item {\textbf{Database used}: MySQL}.
 \item {\textbf{Design used}: HTML, JavaScript}.
 \item {\textbf{Operating System}: Window 11}.
 \item {\textbf{Software used}: XAMPP}.
\end{itemize}

\section{Hardware Requirement Specification} 

\begin{itemize}
 \item {\textbf{Installed Memory} : 2GB or Higher}
 \item {\textbf{Processor}: 1GHz or Higher}.
 \item {\textbf{Hard Disk Space}: 16GB availability }.
 \item {\textbf{Display}: Standard outpout display}.
 \
\end{itemize}
\chapter{System Design}
\section{ER Diagram} 
Figure:\ref{fig:tailor_database.drawio.png} shows the ER diagram of Tailor's Database.
\\
\\
\\
\\
\begin{figure}[h]
 \centering
 \includegraphics[width=1\textwidth]{tailor_database.drawio.png}
 \caption{ER diagram}
 \label{fig:tailor_database.drawio.png}
\end{figure}
\\
\\
\\
\\
\\
\\
\\
\\
\section{Schema Diagram} 
Figure:\ref{fig:tailor_schema.jpg} shows the Schema Diagram of Tailor's Database.
\\
\\
\\
\begin{figure}[h]
 \centering
 \includegraphics[width=1\textwidth]{tailor_schema.jpg}
 \caption{Schema diagram}
 \label{fig:tailor_schema.jpg}
\end{figure}
\section{Table description} 
\begin{center}
\begin{table}[h!]
\centering
    \begin{tabular}{|c|c|c|c|}
    \hline
      Attributes &Datatype &Constraints&Description  \\
      \hline
      \hline
         CLIENT-ID&INT&PRIMARY KEY&ID of client  \\
         \hline
         CLIENT-NAME&VARCHAR&NOT NULL& Name of the client\\ 
         \hline
         AGE&INT&NOT NULL& Client's age\\ 
         \hline
         GENDER&VARCHAR&NOT NULL&Client's gender\\
         \hline
         ADDRESS&TEXT&NOT NULL&Client's address\\
         \hline
         PH-NO&BIGINT&NOT NULL&Client's phone number\\
         \hline
         EMAIL-ID&VARCHAR&NOT NULL&Client's email id\\
         \hline
         PASSWORD&VARCHAR&NOT NULL&Client's password\\
         \hline
    \end{tabular}
    \caption{Client details .}
\label{table:1}
    \end{table}
\end{center}

\begin{center}
\begin{table}[h!]
\centering
    \begin{tabular}{|c|c|c|c|}
    \hline
      Attributes &Datatype &Constraints&Description  \\
      \hline
      \hline
         SHIRT-ID&INT&PRIMARY KEY&ID of shirt  \\
         \hline
         CLIENT-ID&INT&FOREIGN KEY&ID of client\\
         \hline
         COLLAR&FLOAT&NOT NULL& Collar size\\ 
         \hline
         SLEEVE&FLOAT&NOT NULL& Length of the sleeve\\ 
         \hline
         CHEST&FLOAT&NOT NULL& Chest's measurement\\
         \hline
         NECK&FLOAT&NOT NULL&Neck to shoulder size\\
         \hline
         SHOULDER&FLOAT&NOT NULL&Shoulder to shoulder length\\
         \hline
    \end{tabular}
    \caption{Shirt Table.}
\label{table:2}
    \end{table}
\end{center}

\begin{center}
\begin{table}[h!]
\centering
    \begin{tabular}{|c|c|c|c|}
    \hline
      Attributes &Datatype &Constraints&Description  \\
      \hline
      \hline
         PANT-ID&INT&PRIMARY KEY&ID of pant  \\
         \hline
         CLIENT-ID&INT&FOREIGN KEY&ID of client\\
         \hline
         WAIST&FLOAT&NOT NULL& Waist size\\ 
         \hline
         THIGH&FLOAT&NOT NULL& Thigh length\\ 
         \hline
         LENGTH&FLOAT&NOT NULL& Length of the pant\\
         \hline
         KNEE&FLOAT&NOT NULL&Knee size\\
         \hline
         LEG-OPENING&FLOAT&NOT NULL&Width of the cuff\\
         \hline
    \end{tabular}
    \caption{Pant Table.}
\label{table:3}
    \end{table}
\end{center}

\begin{center}
\begin{table}[h!]
\centering
    \begin{tabular}{|c|c|c|c|}
    \hline
      Attributes &Datatype &Constraints&Description  \\
      \hline
      \hline
         ORDER-ID&INT&PRIMARY KEY&ID of order \\
         \hline
         CLIENT-ID&INT&FOREIGN KEY&ID of client\\
         \hline
         SHIRT-TYPE&VARCHAR&NULL& Type of shirt\\ 
         \hline
         PANT-TYPE&VARCHAR&NULL& Type of pant\\ 
         \hline
         ORDER-DATE&DATE&NULL& Date of order\\
         \hline
         DELIVERY-DATE&DATE&NULL&Date of delivery\\
         \hline
         ORDER-TOTAL&DECIMAL&NULL&Total order amount\\
         \hline
    \end{tabular}
    \caption{Orders Table.}
\label{table:4}
    \end{table}
\end{center}

\chapter{Screenshots}

Figure:\ref{fig:HomePage.jpeg} shows the screenshot of home page 
\begin{figure}[h]
 \centering
 \includegraphics[width=0.75\textwidth]{HomePage.jpeg}
 \caption{Home Page}
 \label{fig:HomePage.jpeg}
\end{figure}
\\
\\
\\
Figure:\ref{fig:signup.jpeg} shows the screenshot of sign up page
\begin{figure}[h]
 \centering
\includegraphics[width=0.75\textwidth,height=0.5\textwidth]{signup.jpeg}
 \caption{Sign Up Page}
 \label{fig:signup.jpeg}
\end{figure}
\\
\\
Figure:\ref{fig:PersonalDetails.jpeg} shows the screenshot of personal details page
\begin{figure}[h]
 \centering
 \includegraphics[width=0.75\textwidth]{PersonalDetails.jpeg}
 \caption{Personal Details Page}
 \label{fig:PersonalDetails.jpeg}
\end{figure}
\\
\\
Figure:\ref{fig:Measurements.jpeg} shows the screenshot of measurement details page
\begin{figure}[h]
 \centering
 \includegraphics[width=0.75\textwidth]{Measurements.jpeg}
 \caption{Measurement Details Page}
 \label{fig:Measurements.jpeg}
\end{figure}
\\
\\
Figure:\ref{fig:Shirt_and_Pant.jpeg} shows the screenshot of shirt and pant details page
\begin{figure}[h!]
 \centering
 \includegraphics[width=0.75\textwidth]{Shirt_and_Pant.jpeg}
 \caption{Shirt and Pant Details Page}
 \label{fig:Shirt_and_Pant.jpeg}
\end{figure}
\\
\chapter{Conclusion and Future Scope}

\textbf{Conclusion} 
\begin{spacing}{1.5}
\begin{itemize}
 \item {In conclusion, the Tailor's Database Management System (DBMS) project offers numerous benefits for tailor shops by streamlining operations, improving customer service, and providing valuable insights for business growth. By effectively managing customer information, orders, inventory, and financial data, the DBMS enhances efficiency, accuracy, and customer satisfaction.}
 \item {The project's successful implementation involves careful planning, database design, user-friendly interfaces, and robust security measures. Regular maintenance and support are essential to ensure the system's continued effectiveness.}
\end{itemize}
\textbf{Future Scope} 
\begin{itemize}
 \item {\textbf{Integration with E-commerce}: As online shopping and custom clothing services become more popular, integrating the DBMS with an e-commerce platform can expand the tailor shop's reach and customer base.}

\item{\textbf{Mobile Application}: Develop a mobile app for customers to schedule appointments, track orders, and communicate with the tailor shop conveniently.}

\item{\textbf{Machine Learning and AI}: Implement AI-driven features for trend analysis, size recommendations, and personalized style suggestions based on customer data.}

\item{\textbf{Multi-location Support}: If the tailor shop expands to multiple locations, the DBMS can be extended to support centralized management of all branches.}

\item{\textbf{Customer Feedback and Reviews}: Include features for customers to provide feedback and reviews, helping the tailor shop improve its services and reputation.}.
\end{itemize}
\end{spacing}

\begin{thebibliography}{7}
\addcontentsline{toc}{chapter}{References}
\bibitem{texbook}
Donald E. Knuth (1986) \emph{The \TeX{} Book}, Addison-Wesley Professional.

\bibitem{lamport94}
Leslie Lamport (1994) \emph{\LaTeX: a document preparation system}, Addison
Wesley, Massachusetts, 2nd ed.
\end{thebibliography}


\end{document}