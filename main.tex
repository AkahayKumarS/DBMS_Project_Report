\documentclass[12pt, letter paper]{report}
\usepackage{graphicx} % Required for inserting images
\usepackage[document]{ragged2e}
\usepackage[a4paper, total={6in, 7in}]{geometry}
\setcounter{tocdepth}{6} 
\setcounter{secnumdepth}{6} 
\setcounter{chapter}{0}
\usepackage{blindtext}
\usepackage{layout}
\usepackage[dvipsnames]{xcolor}
\usepackage{biblatex} %Imports biblatex package
\addbibresource{sample.bib} %Import the bibliography file
\usepackage{geometry}
 \geometry{
 a4paper,
 total={170mm,257mm},
 left=27mm,
 top=20mm,
 right=27mm,
 }


\title{DBMS MINIPROJECT REPORT }
\author{Akshay}
\date{August 2023}

\begin{document}
\thispagestyle{empty}
%\maketitle
%cover page 
\begin{center}
\Large\textbf{ ST JOSEPH ENGINEERING COLLEGE\\ }
\textbf{An Autonomous Institution}\\
Affiliated to VTU, Belagavi\\
\textbf{Mangaluru-575028}
\end{center}

\begin{figure}[h]
 \centering
 \includegraphics[scale=1.1]{sjec.jpeg}
 \label{sjeclogo}
\end{figure}

\begin{center}
   \textbf{ MINI PROJECT REPORT ON }
\end{center}
\begin{center}
   \Large\textbf{“TAILOR'S DATABASE MANAGEMENT SYSTEM”}
\end{center}



 
\begin{center}
   \Large \textbf{Submitted By \\}
\end{center}

\begin{center}
\justifying{
\hspace{3.5 cm}Abhik L Salian\hspace{3.5cm}4SO21CS004\\
%\vspace{0.5 cm}
%XXXXXX\hspace{9.5 cm}4SO21CS403
}
\justifying{

\hspace{2.9 cm}Akshaya Kumar S\hspace{2.9 cm}4SO21CS014
}
\end{center}
   \begin{center}
\large \textbf{\\Under the guidance of \\}
\large  \textbf{ \\Ms Pruthvi  M R\\ }
\large  Assistant Professor,\\ Department of CSE
\end{center}

\vspace{0.5cm}
\begin{center}
\large\textbf{\\ DEPARTMENT OF COMPUTER SCIENCE AND ENGINEERING\\ 2022-2023}

\end{center}
\vspace{2cm}
\thispagestyle{empty}
% certificate Page 
\section*{}
\begin{center}
\Large\textbf{\\ ST JOSEPH ENGINEERING COLLEGE}\\
\end{center}
\begin{center}
\textbf{\textit{An Autonomous Institution}}\\

\large \textbf{Vamanjoor, Mangaluru-575028}\\
\large \textbf{ DEPARTMENT OF COMPUTER SCIENCE AND ENGINEERING} 
\end{center}
\vspace{0.1cm}
\begin{center}
\begin{figure}[h]
 \centering
 \includegraphics[scale=0.7]{sjec.jpeg}
 \label{sjeclogo}
\end{figure}
\end{center}
\vspace{-1cm}
\begin{center}
\large\textbf{\textcolor{blue}{CERTIFICATE}}
\end{center}
\begin{center}
\justifying{
\large Certified that the project work entitled \textbf{“Tailor's Database Management System”} carried out by }
\end{center}
\begin{center}
\justifying{
\hspace{3.5 cm}Abhik L Salian\hspace{2.62 cm}4SO21CS004
}
\end{center}
\begin{center}
\justifying{
\hspace{3.5 cm}Akshaya Kumar S\hspace{2.1 cm}4SO21CS014
}
\end{center}

\begin{center}
\justifying{
\par{

\large bonafide students of IV semester students in partial fulfillment for the award of Bachelor of Engineering in Computer Science and Engineering of St Joseph Engineering College during the year 2022-23.It is certified that all corrections/\\suggestions indicated during Internal Evaluation have been incorporated in the report.The project report has been approved as it satisfies the academic requirements in respect of miniproject work .\\\\\\

} 
}
\end{center}
\begin{center}
\justifying{ 
\large\textbf{Ms Pruthvi M R \hspace{5 cm}  Dr Sridevi Saralaya  \\}
\hspace{1cm}\large\textbf{  Project Guide \hspace{6.7 cm} HOD-CSE 
}
}

\end{center}
\begin{center}
  
\large\textbf{ \\EXTERNAL VIVA}
\end{center}
\justifying{ 
\large\textbf{
NAME OF THE EXAMINER  \hspace{4 cm}  SIGNATURE \\\\ 1.\\\\2.}}

\thispagestyle{empty}
\newpage
\chapter*{\centering Acknowledgment}
\justifying{
\addcontentsline{toc}{chapter}{\numberline{}Acknowledgment}
% \linespread
  \large The satisfaction and euphoria that accompanies the successful completion of any task would be incomplete without mentioning the people who made it possible, whose constant guidance and encouragement crowned our efforts with success.\\
    \\
    We take this opportunity to thank those who have helped and motivated us throughout the completion of this project.\\
    \\
    We would like to express our deep and sincere gratitude to our project guide,\textbf{Ms Pruthvi M R}, Assistant Professor, Department of Computer Science and Engineering, for her constant guidance and support, without which this project wouldn’t have been completed successfully. \\
    \\ 
    We owe our great debt to \textbf{Dr Sridevi Saralaya}, Head of the Department of Computer Science and Engineering, for her support and encouragement during the course of development of this project. \\
    \\
    We are immensely grateful to our Principal, \textbf{Dr Rio D’Souza}, our Director, \textbf{Rev. Fr Wilfred Prakash D’Souza}, and Assistant Director \textbf{Rev. Fr Kenneth Rayner Crasta} for their support and encouragement. \\
    \\
    We extend our gratitude to the entire faculty and the staff of the Department of Computer Science and Engineering, SJEC, for their advice, kind co-operation and assistance throughout the academic year.\\
    \\
    Lastly, we would like to express our heartfelt appreciation towards our classmates and seniors for their guidance and suggestions.
 
    
 }
%/\thispagestyle{empty}
\pagenumbering{roman}
\chapter*{\centering Abstract}

\justifying{
\addcontentsline{toc}{chapter}{\numberline{}Abstract}
Para I	(60 words approx) 	Shall introduce the reader the subject matter of the project work / dissertation. \\
\\
Para II	(100 words approx) 	Shall discuss in brief the developments in the area of work/research so far based on the reference to the literature. Identification of the problem and defining exactly the purpose of the intended work or proposition of the novel solution. \\
\\
Para III	(60 words approx)	Conclusions drawn based on the work. Finally, write the proposition of the scope for future work. \\
\\
Overall 300 Words Approximately (Not More than one  pages) 
}
%/\thispagestyle{empty}



%\section*{\centering TABLE OF CONTENT}
%/\thispagestyle{empty}
\renewcommand{\contentsname}{Table of Contents}
\tableofcontents
\addcontentsline{toc}{chapter}{\numberline{}Table of Contents}
\listoffigures
\addcontentsline{toc}{chapter}{\numberline{}List of Figures}
\listoftables
\addcontentsline{toc}{chapter}{\numberline{}List of Tables}
\newpage
%/\tableofcontents
\thispagestyle{empty}

\chapter{Introduction}
An efficient and well-organized Tailor's Database Management System (DBMS) is the cornerstone of success for modern tailoring businesses. In an industry where precision, customization, and customer satisfaction are paramount, a tailor's DBMS serves as the digital backbone that enables tailors to seamlessly manage their operations, clientele, inventory, and much more.

In this digital age, where precision meets artistry, the Tailor's DBMS is not merely a tool; it's a strategic asset that enables tailors to create exquisite clothing, build enduring customer relationships, and navigate the intricate threads of the fashion industry with precision and finesse. In this introduction, we will delve deeper into the multifaceted world of a Tailor's DBMS, exploring its features, benefits, and the transformative impact it has on the art of tailoring.
\section{Problem Definition} 
To create a database that can store client’s data like client name, phone number, address, and all the measurements so that the tailor can use the data in the database without having to measure again and again.

%\subsection{vvv}
\section{Scope and Importance}
\begin{itemize}
\item \textbf{Customer Management}: A tailor's DBMS can store and organize customer information, including measurements, preferences, contact details, and purchase history. This data can be used to offer personalized services and promotions.
\item \textbf{Order Processing}: The system can streamline the order management process, from taking measurements and fabric selection to tracking the progress of each order. This reduces errors and ensures timely deliveries.
\item \textbf{Data Security}: A DBMS can provide data security features to protect sensitive customer information, ensuring that it's not compromised.
\item \textbf{Competitive Advantage}: Implementing a modern DBMS can give tailors a competitive edge in the fashion industry by offering better services, faster delivery, and a more personalized customer experience.
\end{itemize}
\pagenumbering{arabic}
\chapter{Software Requirement Specification}

\section{Functional Requirement Specification} 
Functional  Requirement Specification
\begin{itemize}
\item Details of operations conducted in every screen
\item Data handling logic should be entered into the system
\item It should have descriptions of system reports or other outputs
\item Complete information about the workflows performed by the system
\item It should clearly define who will be allowed to create/modify/delete the data in the system
\item How the system will fulfill applicable regulatory and compliance needs should be captured in the functional document.
\end{itemize}
\section{Software Requirement Specification}
Software Requirement Specification
\begin{itemize}
 \item {\textbf{Language} : HTML, CSS, Java Script, PHP}
 \item {\textbf{Database used}: MySQL}.
 \item {\textbf{Design used}: HTML, JavaScript}.
 \item {\textbf{Operating System}: Window 11}.
 \item {\textbf{Software used}: XAMPP}.
\end{itemize}

\section{Hardware Requirement Specification} 

\begin{itemize}
 \item {\textbf{Installed Memory} : 2GB or Higher}
 \item {\textbf{Processor}: 1GHz or Higher}.
 \item {\textbf{Hard Disk Space}: 16GB availability }.
 \item {\textbf{Display}: Standard outpout display}.
 \
\end{itemize}
\chapter{System Design}
\section{ER Diagram} 
Figure:\ref{fig:tailor_database.drawio.png} shows the ER diagram of Tailor's Database.
\begin{figure}[h]
 \centering
 \includegraphics[width=1\textwidth]{tailor_database.drawio.png}
 \caption{ER diagram}
 \label{fig:tailor_database.drawio.png}
\end{figure}
\\
\\
\\
\section{Schema Diagram} 
Figure:\ref{fig:tailor_schema.jpg} shows the Schema Diagram of Tailor's Database.
\\
\\
\\
\begin{figure}[h]
 \centering
 \includegraphics[width=1\textwidth]{tailor_schema.jpg}
 \caption{Schema diagram}
 \label{fig:tailor_schema.jpg}
\end{figure}
\section{Table description} 
\begin{center}
\begin{table}[h!]
\centering
    \begin{tabular}{|c|c|c|c|}
    \hline
      Attributes &Datatype &Constraints&Description  \\
      \hline
      \hline
         STUD-ID&INT&PRIMARY KEY&ID of student  \\
         \hline
         STUD-NAME&VARCHAR&NOT NULL& Name of the student\\ 
         \hline
         Email-id&VARCHAR&NOT NULL& Email of the student\\ 
         \hline
    \end{tabular}
    \caption{student details .}
\label{table:1}
    \end{table}
\end{center}

\begin{center}
\begin{table}[h!]
\centering
    \begin{tabular}{|c|c|c|c|}
    \hline
      Attributes &Datatype &Constraints&Description  \\
      \hline
      \hline
         COURSE-ID&INT&PRIMARY KEY&ID of course  \\
         \hline
         COURSE-NAME&VARCHAR&NOT NULL& Name of the course\\ 
         \hline
         DEPT-ID&VARCHAR&NOT NULL& department id\\ 
         \hline
    \end{tabular}
    \caption{course table .}
\label{table:1}
    \end{table}
\end{center}
\chapter{Screenshots}

Figure:\ref{fig:home page.png} shows the screenshot of home page 
\begin{figure}[h]
 \centering
 \includegraphics[width=0.75\textwidth]{home page.png}
 \caption{Home Page}
 \label{fig:home page.png}
\end{figure}

Figure:\ref{fig:SIGNIN.png} shows the screenshot of Sign in  page 
\begin{figure}[h]
 \centering
 \includegraphics[width=0.75\textwidth]{SIGNIN.png}
 \caption{User Signin Page}
 \label{fig:SIGNIN.png}
\end{figure}
\\All the pages and database screen shots need to be added .
\chapter{Conclusion and Future Scope}

\textbf{Conclusion} 
\begin{itemize}
 \item {you can use bullet option to write the conclusion}
 \item {The text in the entries should be minimum 100 words }.
\end{itemize}
\textbf{Future Scope} 
\begin{itemize}
 \item {you can use bullet option to write the Future Scope}
 \item {The text in the entries should be minimum 100 words }.
\end{itemize}

\begin{thebibliography}{7}
\addcontentsline{toc}{chapter}{References}
\bibitem{texbook}
Donald E. Knuth (1986) \emph{The \TeX{} Book}, Addison-Wesley Professional.

\bibitem{lamport94}
Leslie Lamport (1994) \emph{\LaTeX: a document preparation system}, Addison
Wesley, Massachusetts, 2nd ed.
\end{thebibliography}


\end{document}
